\documentclass{article}
\usepackage[utf8x]{inputenc}
\usepackage[T1, T2A]{fontenc}
\usepackage[russian]{babel}
\usepackage{amsmath}
\usepackage{amssymb}
\setlength\parindent{0pt}
\usepackage[parfill]{parskip}
\pagenumbering{gobble}

\begin{document}
В корзине лежит $m$ чёрных шаров и $n$ красных. Вася достаёт из корзины случайный шар и, если он чёрный, то заменяет его на красный, а если он красный, то кладёт его обратно.
Найдите математическое ожидание и дисперсию числа красных шаров в корзине после $k$ итераций этой процедуры. Оба ответа должны быть компактными выражениями (то есть не содержать 
знаков суммирования, многоточий и пр.).
\end{document}
