\documentclass{article}
\usepackage[utf8x]{inputenc}
\usepackage[T1, T2A]{fontenc}
\usepackage[russian]{babel}
\usepackage{amsmath}
\usepackage{amssymb}
\setlength\parindent{0pt}
\usepackage[parfill]{parskip}
\pagenumbering{gobble}

\begin{document}
В королевстве Грок некоторые города соединены двусторонними магическими порталами, причем из каждого города можно попасть в каждый за несколько телепортаций. 
Когда из города $A$ в город $B$ отправляют груз, то по закону стоимость пересылки равна \textit{кратчайшему расстоянию} между $A$ и $B$: минимально возможному 
количеству ребер на пути между $A$ и $B$.

В архиве почтовой службы вы нашли упоминание о диаметре королевства --- то есть о максимально возможном кратчайшем расстоянии между парой вершин, --- а также 
следующий способ его вычисления. Занумеруем все города числами от $1$ до $n$. Выберем в качестве $A_0$ город с номером $1$ и найдем кратчайшее расстояние от него 
до всех остальных городов. Выберем в качестве города $A_1$ наиболее удаленный от $A_0$, среди всех таких выберем город с минимальным индексом. Теперь найдем 
кратчайшее расстояние от города $A_1$ до всех остальных городов, и в качестве $A_2$ выберем наиболее удаленный, а среди таковых город с минимальным номером 
(это может быть снова $A_0$). Далее аналогично построим $A_3$, $A_4$ и так далее до $A_k$ для некоторого $k$. Теперь в качестве диаметра выберем максимальное 
расстояние между всеми парами $(A_i, A_{i+1})$ для $i$ от $0$ до $k-1$. Приведите пример, который покажет, что такое решение не сработает, как бы мы ни 
выбирали значение параметра $k$. В вашем королевстве должно быть не более $10$ городов, соединенных не более чем $100$ порталами.
\end{document}