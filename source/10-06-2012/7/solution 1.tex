\documentclass{article}
\usepackage[utf8x]{inputenc}
\usepackage[T1, T2A]{fontenc}
\usepackage[russian]{babel}
\usepackage{amsmath}
\usepackage{amssymb}
\setlength\parindent{0pt}
\usepackage[parfill]{parskip}
\pagenumbering{gobble}

\begin{document}
Пусть квадратичная форма записана в базисе $e_1, e_2, \ldots e_n$. Понятно, что если ответ на какой-то вопрос будет неположительным, то форма не положительно определенная. 
Задав $\frac{n(n+1)}{2}$ вопросов можно найти все коэффициенты (например, сначала находим диагональные, затем все остальные) и определить, является ли форма положительно определенная. Докажем, что 
за меньшее число вопросов сделать мы этого не сможем.

Каждый вопрос дает нам линейное уравнение на коэффициенты. Если мы задали вопросов меньше, чем число коэффициентов, какие-то 
коэффициенты будет неопределены. Пусть неопределен диагональный коэффициент $a_{jj}$. Тогда ответ на вопрос $Q(e_{j}) = a_{jj}$ может 
быть отрицательным, а значит мы не можем быть уверены в положительной определенности. Пусть теперь все диагональные коэффициенты известны и положительны, а 
некоторый недиагональный коэффициент $a_{ij}$ неопределен. Тогда ответ на вопрос $Q(e_i + e_j) = a_{ii} + a_{ij} + a_{jj}$ может быть отрицательным.
\end{document}
