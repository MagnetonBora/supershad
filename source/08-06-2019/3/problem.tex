\documentclass{article}
\usepackage[utf8x]{inputenc}
\usepackage[T1, T2A]{fontenc}
\usepackage[russian]{babel}
\usepackage{amsmath}
\usepackage{amssymb}
\setlength\parindent{0pt}
\usepackage[parfill]{parskip}
\pagenumbering{gobble}

\begin{document}
Алёна очень любит алгебру. Каждый день, заходя на свой любимый алгебраический форум, она с вероятностью $\frac14$ 
находит там новую интересную задачу про группы, а с вероятностью $\frac{1}{10}$ интересную задачку про кольца. 
С вероятностью $\frac{13}{20}$ новых задач на форуме не окажется. Пусть $X$ --- это минимальное число дней, за
которые у Алёны появится хотя бы одна новая задача про группы и хотя бы одна про кольца. Найдите распределение 
случайной величины $X$. В ответе должны участвовать только компактные выражения (не содержащие знаков суммирования,
многоточий и пр.).
\end{document}
