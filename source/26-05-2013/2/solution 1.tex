\documentclass{article}
\usepackage[utf8x]{inputenc}
\usepackage[T1, T2A]{fontenc}
\usepackage[russian]{babel}
\usepackage{amsmath}
\usepackage{amssymb}
\setlength\parindent{0pt}
\usepackage[parfill]{parskip}
\pagenumbering{gobble}

\begin{document}
Будем решать задачу от противного. Предположим, что у любой пары подмножеств из условия мощность симметрической разности больше двух. 
Возьмем тогда некоторую пару различных подмножеств $A$ и $B$. Назовем $f(A)$ и $f(B)$ все возможные подмножества, у 
которых мощность симметрической разности с ними равна одному (эти подмножества не входят в $100$). Ясно, что $|f(A)| = |f(B)| = 10$.
Докажем, что $f(A)$ и $f(B)$ не пересекаются. 
В самом деле, пусть есть некоторое подмножество $C$, которое принадлежит $f(A)$ и $f(B)$ одновременно, то есть
$|C \bigtriangleup A| = 1$ и $|C \bigtriangleup B| = 1$. Но тогда $|A \bigtriangleup B| = 2$, чего быть не может.
Далее, поскольку для любой пары множеств $f(A)$ и $f(B)$ не пересекаются, то каждому множеству (из $100$) можно поставить в соответствие 
$10$ других множеств (не из $100$). Таким образом, общее число подмножеств должно быть как минимум $11 \times 100 = 1100$. 
Но а нас есть только $2^{10} = 1024$ подмножества.

\textit{Примечание.} Данное решение можно сформулировать более красиво, если рассмотреть представления подмножеств в $10$-мерном бинарном 
пространстве $\{0, 1\}^{10}$, где стоит $1$ если элемент принадлежит подмножеству и $0$ в противном случае. Расстоянием между 
элементами пространства назовем расстояние Хэмминга, то есть число позиций, в которых соответствующие символы различны. 
Тогда единичная сфера в таком пространстве состоит из $10$ точек, а если мы поместим в это пространство $100$ точек, какие-то 
две сферы вокруг них должны пересечься. Центры таких сфер соответствуют подмножествам, у которых симметрическая разность имеет мощность не более 
двух.

\end{document}
