\documentclass{article}
\usepackage[utf8x]{inputenc}
\usepackage[T1, T2A]{fontenc}
\usepackage[russian]{babel}
\usepackage{amsmath}
\usepackage{amssymb}
\setlength\parindent{0pt}
\usepackage[parfill]{parskip}
\pagenumbering{gobble}

\begin{document}
Запишем многочлен от двух переменных в общем случае:
$$f(x, y) = \sum a_{nm} x^n y^m$$
Распишем интеграл по окружности:
$$\oint\limits_{\{x^2+y^2=R^2\}} f(x,y) ds = \int\limits_0^{2\pi} \sum a_{nm} (R\cos \varphi)^n (R\sin \varphi)^m d\varphi = $$
$$ = \sum a_{nm} R^{n+m} \int\limits_0^{2\pi} (\cos \varphi)^n (\sin \varphi)^m d\varphi = 0.$$
Если мы хотим, чтобы это равенство было выполнено для любого $R$, коэффициенты при различных степенях $R$ должны быть равны нулю.
Рассмотрим более подробно интеграл 
$$c_{nm} = \int\limits_0^{2\pi} (\cos \varphi)^n (\sin \varphi)^m d\varphi.$$

Заметим, что $c_{nm} = c_{mn}$, поскольку нам не важно в какой точке окружности мы начнем интегрировать (более формально: сделаем замену $\varphi = \frac{\pi}{2} - \alpha$ 
и прибавим $\frac{3\pi}{2}$ к каждому пределу интегрирования).

У нас есть три случая:\\
1) $n$ и $m$ четные. Тогда подынтегральное выражение всегда неотрицательно, поэтому $c_{nm} > 0$.\\
2) $n$ и $m$ имеют разную четность. Поскольку $c_{nm} = c_{mn}$, достаточно рассмотреть случай, в котором $n$ --- четное, а $m$ --- нечетное. Тогда 
подыинтегральное выражение антисимметрично относительно точки $\pi$, поэтому $c_{nm} = 0$.\\
3) $n$ и $m$ нечетные. Тогда интеграл можно разбить на два участка: от $0$ до $\pi$ и от $\pi$ до $2\pi$. На каждом из этих участков подынтегральное выражение антисимметрично 
относительно центра ($\frac{\pi}{2}$ и $\frac{3\pi}{2}$ соответственно), поэтому $c_{nm} = 0$.

(Если эти рассуждения не совсем понятны, попробуйте нарисовать графики $\sin x$, $\cos x$, $\sin^2 x$ и $\cos^2 x$. Более старшие степени только деформируют эти графики, 
не меняя указанных симметрий. Попробуйте понять, как выглядят графики их произведений.)

Таким образом $c_{nm}$ не равен нулю, только когда $n$ и $m$ четные. Тогда в базисный набор мы сразу можем включить все одночлены $x^n y^m$, у которых 
$n$ и $m$ имеют разную четность либо оба нечетные. Но это еще не все возможные базисы. Несмотря на то, что для четных $n$ и $m$ коэффициенты $c_{nm}$ 
больше нуля, их линейная комбинация при одинаковой степени все равно может дать нуль. Это соответствует линейному ограничению на коэффициенты для 
четных степеней $R$, а значит уменьшает число базисных одночленов на $1$ от всех возможных для каждой такой степени.

Посчитаем число одночленов от двух переменных степени не выше $s$. Есть $1$ одночлен степени $0$, $2$ одночлена степени $1$ и так далее. Общее число 
одночленов равно 
$$1 + 2 + \cdots + (s+1) = \frac{(s+1)(s+2)}{2}.$$
Число возможных четных степеней $R$ равно $\lfloor \frac{s}{2} \rfloor + 1$. Таким образом, размерность подпространства $V$ равна:
$$\text{dim}\, V = \frac{(s+1)(s+2)}{2} - \lfloor \frac{s}{2} \rfloor - 1 = \frac{s(s+3)}{2} - \lfloor \frac{s}{2} \rfloor.$$

Для случая $s = 2013$ получим $\text{dim}\, V = 2\,028\,098$.

\textit{Примечение.} Выпишем явно базисные одночлены для первых $s$. Поскольку для очередного $s$ можно использовать предыдущий базис, будем выписывать 
только новые одночлены:
\begin{table}[h!]
\begin{tabular}{cccc}
$s$ & Размерность $V$ & С разной четностью и нечетные  & Четные\\
\hline
$1$ & $2$ & $x$, $y$ & \\
$2$ & $4$ & $xy$ & $x^2 - y^2$ \\
$3$ & $8$ & $x^3$, $y^3$, $x^2y$, $xy^2$ & \\
$4$ & $12$ & $x^3y$, $xy^3$ & $x^4 - 3x^2y^2$, $y^4 - 3x^2y^2$ \\
$5$ & $18$ & $x^5$, $y^5$, $x^4y$, $xy^4$, $x^2y^3$, $x^3y^2$ & \\
\end{tabular}
\end{table}

Покажем, как получился, например, базис для $n=4$. При $R^4$ в разложении интеграла от произвольного многочлена стоит следующее выражение:
$$a_{40} c_{40} + a_{04} c_{04} + a_{31} c_{31} + a_{13} c_{13} + a_{22} c_{22} = a_{40} c_{40} + a_{04} c_{04} + a_{22} c_{22}.$$
Условие на коэффициенты:
$$a_{40} c_{40} + a_{04} c_{04} + a_{22} c_{22} = 0.$$
Подставим в разложение
$$a_{40} x^4 + a_{04} y^4 + a_{22} x^2 y^2$$
получившееся условие:
$$a_{40} \left( x^4 -  \frac{c_{40}}{c_{22}} x^2y^2 \right) + a_{04} \left( y^4 - \frac{c_{04}}{c_{22}} x^2y^2 \right).$$
Нетрудно посчитать
$$c_{40} = c_{04} = \frac{3\pi}{4}, \;\;\; c_{22} = \frac{\pi}{4}.$$
Поэтому выражение превращается в
$$a_{40} (x^4 - 3x^2y^2) + a_{04} (y^4 - 3x^2y^2).$$
\end{document}
