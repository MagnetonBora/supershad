\documentclass{article}
\usepackage[utf8x]{inputenc}
\usepackage[T1, T2A]{fontenc}
\usepackage[russian]{babel}
\usepackage{amsmath}
\usepackage{amssymb}
\setlength\parindent{0pt}
\usepackage[parfill]{parskip}
\pagenumbering{gobble}

\begin{document}
Пусть $G(t)$ --- производящая функция последовательности:
$$G(t) = \sum_{k=0}^\infty x_k t^k.$$
Умножим каждое равенство в условии на $t^0, t^1, \ldots, t^k, \ldots$:
\begin{align*}    
t^0 x_0 &= 0\\
t^1 x_1 &= t\\
\cdots&\\
t^k x_k &= t^n \left( \frac{x_{k-1} + (k-1)x_{k-2}}{k} \right)\\
\cdots&
\end{align*}

Просуммируем эти равенства и возьмем производную:
$$G'(t) = 1 + G(t) + 2tG(t) + t^2G'(t) - tG(t).$$
Решаем дифференциальное уравнение на $G(t)$:
$$G'(t)(t^2 - 1) + G(t)(t+1) = -1.$$
$$[G(t)(t-1)]'(t+1) = -1.$$
$$G(t) = \frac{C}{1-t} + \frac{\ln (t+1)}{1-t}.$$
Поскольку $G(0) = 0$:
$$G(t) = \frac{\ln (t+1)}{1-t}.$$
Теперь мы можем найти члены последовательности, посмотрев на разложение $G(t)$ в ряд Тейлора:
$$G(t) = (t - \frac{t^2}{2} + \frac{t^3}{3} - \cdots)(1 + t + t^2 + \cdots).$$
Тогда
$$x_n = \sum_{k=1}^n \frac{(-1)^{k+1}}{k}.$$
Поскольку 
$$\ln(t + 1) = \sum_{k=1}^\infty \frac{(-1)^{k+1}t^k}{k},$$
понятно, что
$$\lim_{n \to \infty} x_n = \ln 2.$$
\end{document}
